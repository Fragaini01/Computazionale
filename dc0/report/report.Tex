\documentclass[a4paper,11pt,rmp,superscriptaddress]{revtex4}
\usepackage{latexsym} 
\usepackage{helvet}
\usepackage{times}
\usepackage[applemac]{inputenc}
\usepackage{setspace}
\usepackage{color}
\usepackage{hyperref}
\usepackage{fancyhdr}
\usepackage{graphicx}
\usepackage{wrapfig}
\usepackage{ulem} 
\usepackage{amsmath}
\usepackage{amssymb}
\usepackage[margin=2cm]{geometry}
\renewcommand{\baselinestretch}{1} 
\bibliographystyle{authordate1}

\begin{document}

\title{Comparing Global Temperature Datasets: Evidence for a Significant Increase in Temperature in the Last 50 Years}
\author{Francesco Ragaini}
\begin{abstract}

In the last centuries the global temperature has changed and many different survey have collected data about that.\\
This report intends to compare data from five different observer and to test if the anomaly in the last fifty years can be considered as a statistical fluctuation

To achive this goals has been collected datasets from Nasa, HadCRUT, Japan metereological office, Berkeley earth, Ncdc Noaa, 
this datasets has been compared with a Kalmogorv-Smirnov test, than all of that has been split in data before and after 1975 and compered 
again with KS test.

The Kalmogorv-Smirnov test on the entire datasets has reported positive results (\ref*{fig:1}), conversely the K.S. test 
on the split dataset has reported negative results.

This result suggest that all the datasets analyzed describe the same effect and the distribution of the anomaly after the 1975 and 
before 1975 have different distributions. Because of that results can be concluded that all survey registered the same effect
and in the last fifty years the temperature as increased.
\end{abstract}

\maketitle
\section*{Data analisis}
All the surveys analyzed in this study have released their datasets on their respective websites (see bibliography).
However, each dataset considers a different temperature as the zero point. Therefore, 
the first step is to adjust the datasets by applying an offset to set the zero point at the same temperature. 
Only after this correction, K.S. tests can be applied to the datasets.

\section*{Data comparison}
\begin{figure}[h!]
    \centering
    \includegraphics[width=0.8\textwidth]{"/home/francesco/Codici/Computazionale/dc0/Sovrapposizione_Osservatori.png"}
    \caption[short]{All the distributions are describing a similar phenomen. P value of the K.S. test (Table in figure) is bigger than 0.05, 
    so the distributions can be superimposed.}
    \label{fig:1}
\end{figure}
As shown in \ref*{fig:1}, all anomaly datasets exhibit a high degree of consistency with each other, 
as evidenced by the K.S. p values reported in the table. 
While the Japan Met. dataset appears to be the most different from the others, 
its p value with respect to all the other datasets is still greater than 0.05. 
Overall, these findings suggest a high level of agreement among the different surveys 
regarding the measurement of global temperature anomalies.

\begin{figure}[h!]
    \centering
    \includegraphics[width=1\textwidth]{"/home/francesco/Codici/Computazionale/dc0/Pre&Post.png"}
    \caption[short]{Anomalies before and after 1975 are not consistent, p value in K.S. test are a lot smaller than 0.05}
    \label{fig:2}
\end{figure}
In \ref*{fig:2} is shown that all datasets agree that the distributions of anomalies before and after 1975 are not consistent, 
as every p-value is under 0.05.


\section*{Conclusions}
Because of first set of K.S. test, (\ref*{fig:1}) is possibile to assert that different survey are misuring similar distributions,
so can be concluded that all of that are misuring the same effect. Japan Met. data are the most different to the others, this could 
appen because Japan Met. zero period was the most different 

In second set of K.S. test, (\ref*{fig:2}) is shown that the anomalies in the last fifty years can't be considered as statistical fluctuation.
Because of that could be interesting to search for a trend in the last fifty years of data.


\section*{Bibliography}
Nasa:\\
\href{https://data.giss.nasa.gov/gistemp/graphs_v4/graph_data/Global_Mean_Estimates_based_on_Land_and_Ocean_Data/graph.txt}{Dataset}\\
Base: 1951-1980\\

HadCRUT:\\
\href{https://www.metoffice.gov.uk/hadobs/hadcrut5/data/current/download.html} {Dataset}\\
Base: 1961-1990\\

Ncdc Noaa\\
\href{https://www.ncdc.noaa.gov/cag/global/time-series/globe/land_ocean/12/12/1880-2023/data.csv}{Dataset} \\
Base: 1901-2000\\

Japan metereological office\\
\href{https://ds.data.jma.go.jp/tcc/tcc/products/gwp/temp/list/csv/year_wld.csv}{Dataset} \\
Base: 1991-2020\\

Berkeley\\
\href{https://ds.data.jma.go.jp/tcc/tcc/products/gwp/temp/list/csv/year_wld.csv}{Dataset} \\
Base: 1951-1980\\

\end{document}