\documentclass[a4paper,11pt,rmp,superscriptaddress]{revtex4}
\usepackage{latexsym} 
\usepackage{helvet}
\usepackage{times}
\usepackage[applemac]{inputenc}
\usepackage{setspace}
\usepackage{color}
\usepackage{hyperref}
\usepackage{fancyhdr}
\usepackage{graphicx}
\usepackage{wrapfig}
\usepackage{ulem} 
\usepackage{amsmath}
\usepackage{amssymb}
\usepackage[margin=2cm]{geometry}
\renewcommand{\baselinestretch}{1} 
\bibliographystyle{authordate1}

\begin{document}
\author{Francesco Ragaini}

\title{Comparing Global Temperature Datasets: Evidence for a Significant Increase in Temperature in the Last 50 Years}
\begin{abstract}

On the last centuries the global temperature has changed and many different surveys have collected data about that.\\
This report intends to compare data from five different observers and to test if the anomaly in the last fifty years 
can be considered as a statistical fluctuation.
To achive this goals has been collected datasets from Nasa, HadCRUT, Japan Metereological office, Berkeley earth, Ncdc Noaa, 
this datasets has been compared with a Kalmogorv-Smirnov test, than all of that has been split in data before and after 1975 and compered 
again with KS test.
The Kalmogorv-Smirnov test on the entire datasets has reported positive results (figure \ref*{fig:1}), conversely the K.S. test 
on the split dataset has reported negative results (figure \ref*{fig:2}).
This result suggest that all the datasets analyzed describe the same effect and the timeseries of the anomalies after the 1975 and 
before 1975 have different distributions. According to this results all surveys registered the same effect
and in the last fifty years the temperature as increased.
\end{abstract}

\maketitle
\section*{Data analysis}
All the surveys analyzed in this study have released their datasets on their respective websites (see bibliography). 
The datasets are organized as dataseries of annual anomalies: discrepancy of annual global temperature from a certain baseline, 
considered as zero point.
However, each dataset considers a different temperature as the zero point. Therefore, 
the first step is to adjust the datasets by applying an offset to set the zero point at the same temperature. 
Only after this correction, Kalmogorv-Smirnov tests can be applied to the datasets.

\section*{Data comparison}
\begin{figure}[h!]
    \centering
    \includegraphics[width=0.8\textwidth]{"/home/francesco/Codici/Computazionale/dc0/Sovrapposizione_Osservatori.png"}
    \caption[short]{\textbf{All surveys's datasets are consistent with the others by K.S. test.} The plot rappresent normalized histogram with 
    annual anomalies divided by survey. In the table are reported the p values.} 
    \label{fig:1}
\end{figure}
As shown in figure \ref*{fig:1}, all anomaly datasets exhibit a high degree of consistency with each other, 
as evidenced by the K.S. p values reported in the table. 
While the Japan Met. dataset appears to be the most different from the others, 
its p value respect to all the other datasets is still greater than 0.05. 
Overall, these findings suggest a high level of agreement among the different surveys 
regarding the measurement of global temperature anomalies.

\begin{figure}[h!]
    \centering
    \includegraphics[width=0.8\textwidth]{"/home/francesco/Codici/Computazionale/dc0/Pre&Post.png"}
    \caption[short]{\textbf{Anomalies before and after 1975 are not consistent.} Each plot is an histogram that 
    represent anomalies distributions before and after 1975. In the table are reported the p values by K.S. tests.}
    \label{fig:2}
\end{figure}
Figure \ref*{fig:2} shows that all datseries before and after 1975 are not compatible by K.S. tests,
as every p-value is under 0.05. The year 1975 has been chosen beacouse it shows a significant change in trend.


\section*{Conclusions}
According to the first result, (fig \ref*{fig:1}) is possibile to assert that different surveys are misuring similar distributions,
so can be concluded that all of that are misuring the same effect. Japan Met. data are the most different to the others, this could 
appen because Japan Met. zero period was the most different. 

In the second claim, (fig \ref*{fig:2}) is shown that the anomalies in the last fifty years can't be considered as statistical fluctuation.
According to that result can be interesting to search for a trend in the last fifty years of data.

\section*{Bibliography}


\begin{tabular}{ c  c  c  c  c }
    Nasa    &   HadCRUT     &   Japan Metereological office     &   Ncdc Noaa   &   Berkeley \\
    \href{https://data.giss.nasa.gov/gistemp/graphs_v4/graph_data/Global_Mean_Estimates_based_on_Land_and_Ocean_Data/graph.txt}{Nasa dataset} & 
    \href{https://www.metoffice.gov.uk/hadobs/hadcrut5/data/current/download.html} {HadCRUT dataset} &
    \href{https://ds.data.jma.go.jp/tcc/tcc/products/gwp/temp/list/csv/year_wld.csv}{Japan Met. dataset} &
    \href{https://www.ncdc.noaa.gov/cag/global/time-series/globe/land_ocean/12/12/1880-2023/data.csv}{Noaa dataset} &
    \href{https://ds.data.jma.go.jp/tcc/tcc/products/gwp/temp/list/csv/year_wld.csv}{Berkeley dataset} \\
    Base: 1951-1980      &      Base: 1961-1990      & Base: 1991-2020      & Base: 1901-2000    & Base: 1951-1980
\end{tabular}
\end{document}