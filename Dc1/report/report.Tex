\documentclass[a4paper,11pt,rmp,superscriptaddress]{revtex4}
\usepackage{latexsym}
\usepackage{helvet}
\usepackage{times}
\usepackage[applemac]{inputenc}
\usepackage{setspace}
\usepackage{color}
\usepackage{hyperref}
\usepackage{fancyhdr}
\usepackage{graphicx}
\usepackage{wrapfig}
\usepackage{ulem}
\usepackage{amsmath}
\usepackage{amssymb}
\usepackage[margin=2cm]{geometry}
\renewcommand{\baselinestretch}{1}
\bibliographystyle{authordate1}
\begin{document}
\title{Global warming on Twitter: examining the relationship between social media discourse and political votes in Italy, 
France and Germany}
\author{Francesco Ragaini}

\begin{abstract}

    In the last decade, climate change has gained significant attention, with numerous surveys and studies 
    published on the potential disasters associated with it.\\
    This report was conducted to investigate whether there has been an increase in attention towards global warming in Italy, 
    France and Germany and whether this increased attention has influenced political voting patterns in the last two elections in 
    each state.\\
    Twitter was chosen as the primary source of data collection due to its popularity and the fact that tweet data is publicly available.\\ 
    The number of tweets mentioning global warming was counted in each country for every year between 2017 and 2021 and 
    divided by the number of users per year.
    The percentage of tweets about global warming was then calculated for the year prior to the election from the most 
    voted parties in the election.
    This procedure was repeated for the two most recent elections in each of the selected countries.\\
    Next, a correlation was analyzed between the percentage of tweets related to global warming and the political results for each party. 
    This test was conducted on both the total election results and the results of people under 35 years old.\\
    The results showed that the number of tweets on active users increased in the selected period for each country (see Figure 1). 
    Furthermore, the percentage of tweets on global warming increased between the first and second election in each country (see Figure 2).
    However, the correlation between the percentage of tweets and the total results did not increase in Italy, 
    and was only greater than 0.8 in two cases (see Figure 3). 
    On the other hand, the correlation between the percentage of tweets and the results of people under 35 was higher in 
    each election than the correlation between the percentage of tweets and the total results.\\

\end{abstract}


\maketitle

\end{document}