\documentclass[a4paper,11pt,rmp,superscriptaddress]{report}
\usepackage{latexsym}
\usepackage{helvet}
\usepackage{times}
\usepackage[applemac]{inputenc}
\usepackage{setspace}
\usepackage{color}
\usepackage{hyperref}
\usepackage{fancyhdr}
\usepackage{graphicx}
\usepackage{wrapfig}
\usepackage{ulem}
\usepackage{amsmath}
\usepackage{amssymb}
\usepackage{multicol}
\usepackage{comment}
\usepackage[margin=1.5cm]{geometry}
\renewcommand{\baselinestretch}{1}
\bibliographystyle{authordate1}
\begin{document}
\author{Francesco Ragaini}
\title{Analyzing the relationship between house prices and key variables in Italy: Cross-Correlation and Granger Causality approach}
\date{}
\twocolumn[\nopagebreak
\maketitle
\begin{abstract}
The housing market is a crucial aspect of any country, and understanding the factors that influence house prices is crucial for 
policymakers and investors. The objective of this report is to determine if there is a significant cross-correlation and Granger 
causality between a set of variables, namely interest rate, population, building permits, employed individuals, GDP, 
and house price in Italy from 2000 to 2020. In order to ensure comparability between the variables, their mean values in 2010 
were set to 100, thus placing them on the same scale.
To begin, the time series data was tested for stationarity using the Dickey-Fuller test (DF test) to determine whether the series is 
stationary. Becuase of negative results by DF test, the data was then split into three periods based on the trends in house prices 
and detrended. Than DF test was again applied to each timeseries.After that a lagged cross correlation was computed between house prices 
and each of the other variables.
Next, VAR models were constructed for each time series to determine the optimal lag that minimizes the Akaike Information Criterion (AIC). 
Granger causality tests were then conducted on each series against house prices using the optimal lag.
DF tests reveals (fig:\ref*{fig:1}) that all time series variables, are stationary only after detrending, 
expected for population timseries that was stationarity even before detrending. 
The cross-correlation analysis (fig:\ref*{fig:2}) shows an increase when the variable precedes house prices, but this pattern is only 
observed for interest rate.
For the other variables, the cross-correlation is nearly symmetric with a maximum value at zero lag for employees, DCP, and building 
permits, and a minimum value for population. The maximum values for cross-correlation are below 0.4, except for interest rates, which is 
close to 0.6.
The Granger causality test indicates (fig:\ref*{fig:3}) that only the interest rate variable has a p-value of less than 0.05 and 
the optimal between interest rate and house price lag is also the one that perform a hightest cross correlation.
The result of DF tests to detrended data permits to refuse null hypothesis of non-stationarity 
and enable cross-correlation and Granger causality tests.
The cross-correlation analysis indicates a moderate correlation between interest rates and house prices,
with a lead time of two quarters. For other variables, the test is inconclusive and suggests a serch for non-linear relationship.
The Granger causality test shows a causal relationship between interest rates and house prices, but not for other variables.
In summary, the report concludes that interest rates have a causal influence on house prices in Italy,
while other variables may have a more complex relationship that requires further analysis.\\
\end{abstract}
]

\section*{Introduction}
The housing market plays a vital role in the economy of a country, as it impacts several sectors such as construction, 
finance, and real estate. However, predicting changes in house prices is a complex task due to the interplay of various factors, 
including economic conditions, population demographics, and housing policies. 

In this context, this report intends to investigate whether specific variables can be considered causal factors that influence house 
prices in Italy. By employing statistical tests such as the Dickey-Fuller test and the Granger causality test, the study examines 
the relationship between key factors such as GDP, interest rates, new building permits, population, the number of workers, 
and the trends in house prices over time. The findings of the analysis will shed light on the most significant factors driving changes 
in house prices in Italy.

\section*{Methods}
The primary purpose of this report is to investigate whether a noteworthy cross-correlation and Granger 
causality exists between a specified group of variables, namely interest rate, population, building permits, 
employed individuals, GDP, and house price in Italy from 2000 to 2020. All of this series are online (\nameref{section:Bibliography}) 
as a quarterly divided timeseries. To ensure a fair and meaningful comparison between the variables, 
their mean values in 2010 were normalized to 100, effectively placing them on the same scale and reducing the impact of any potential 
outliers. By doing so, the study aims to identify any significant patterns or relationships that could provide insights into the 
determinants of house prices in Italy over the past two decades.

In order to detrend all time series, has been choose to divide the time series in three periods based on maximum and minimum 
of house prices timseries.

The Dickey-Fuller test is a statistical test used to determine if a time series is stationary or not. 
It is based on the null hypothesis that the time series  is non-stationary. By calculating a test statistic, the Dickey-Fuller test 
compares the observed time series with an expected distribution to determine whether the null hypothesis can be rejected. 


Cross-correlation is a statistical measure that examines the relationship between two time series by measuring their similarity or 
dissimilarity over time. 
It computes the correlation between the two series as a function of the time lag between them. To compute cross correlation timeseries
must be stationary.

Granger causality is a statistical concept that investigates whether past values of one variable can predict future values of another 
variable. The idea is that if one variable Granger causes another, then the former provides useful information for forecasting the latter.
The Granger causality test can be performed using a vector autoregression (VAR) model, 
which is a time series model that includes lags of variable as predictors. The best number of lags can be 
determined using the Akaike information criterion (AIC), which is a measure of the relative quality of a statistical model.
Null hypothesis in Granger model is that one variable does not causes the other.
In order to apply Grenger causality test all timeseries must be stationary.

\section*{Results}
Figure \ref*{fig:1} displays the time series before and after detrending. The table shows that after detrending, 
all p-values for the DF test are below 0.05, enabling the use of cross-correlation and Granger causality tests. 
Even before detrending, the population time series had a p-value below 0.05 in the DF test.

\begin{figure}[h!]
    \includegraphics[height = 14cm , width = 9cm]{"/home/francesco/Codici/Computazionale/DC2/Italia/ripuliti/Linearizzati.png"}
    \caption[short]{\textbf{Only after detrending all timeseries are stationary.} Figure shows timeseries before and after detrending, 
    in table are reported p values by DF test: all of that are under 0.05 after detrending.}
    \label{fig:1}
\end{figure}

Figure \ref*{fig:2} depicts the lagged cross-correlation between the variables and house prices. The final chart highlights the
highest cross-correlation values for each variable. Only the interest rate shows a correlation greater than 0.5. Interestingly,
for all variables except interest rate, the cross-correlation is nearly symmetric for positive and negative lagged time and
reaches its maximum value at zero lag for employees, DCP, and building permits, while population has the lowest correlation.
However, none of these correlations have an absolute value greater than 0.4.

\begin{figure}[h!]
    \includegraphics[height = 12cm , width = 9cm]{"/home/francesco/Codici/Computazionale/DC2/Italia/ripuliti/CrossCorreletion.png"}
    \caption[short]{\textbf{The lagged cross-correlation increases only when interest rate precedes house price, as it is the only 
    variable that correletion with house prices exceeds a value of 0.5.} The figure displays the cross-correlation over timelags, with the left side indicating 
    when the variable precedes house price and the right side indicating when house price precedes the variable. 
    The last plot reports the maximum cross-correlation for each variable, and only the cross-correlation between interest rate and 
    house price exceeds value of 0.5, while the other are smaller than 0.4.}
    \label{fig:2}
\end{figure}

Figure \ref*{fig:3} displays the AIC for the VAR model over the maximum lags for variables, and the p-values of the 
Granger test for the best lags chosen as the one that minimizes AIC. Only the Granger's p-values for the VAR model based on 
interest rate are under 0.05. All the other values are bigger than 0.5.

\begin{figure}[h!]
    \includegraphics[height = 12cm , width = 9cm]{"/home/francesco/Codici/Computazionale/DC2/Italia/ripuliti/BestLag.png"}
    \caption[short]{\textbf{Only for interest rate Granger p value for the best lag is under 0.05.} In plot is shown value of AIC over 
    max lag for the VAR model, in the last plot the Granger p value for the best lag, choosed using the Aikakie Information Criterion.
    Only rate interest has a p value under 0.05, while the others are bigger than 0.5.}
    \label{fig:3}
\end{figure}

\section*{Conclusions}
The DF tests for detrended series (fig:\ref*{fig:1}) allow to reject the null hypothesis of non-stationarity, 
thereby enabling the computation of cross-correlation and Granger causality tests. 

The results (fig:\ref*{fig:2}) of the cross-correlation analysis indicate a moderately strong correlation between interest rates and house prices, 
but only when interest rates lead house prices by two quarters. For the other variables, the test is inconclusive but suggests 
that a linear cross-correlation may not be the most appropriate approach for analyzing their relationship with house prices.

The Granger causality test produced results similar to the cross correlation analysis, (fig:\ref*{fig:3}) indicating a significant causal relationship 
between interest rate and house prices with a lag of two quarters refusing the null hypothesis of non-causality. 
However, the test did not reject the null hypothesis of non-causality for the other variables, 
and the p-values suggest that a more complex model than a VAR may be necessary to fully describe their 
relationship with house prices.

In summary, this report has found that interest rates can be considered causal factors that influence house prices in Italy, with a lag of two quarters. 
On the other hand, the results of the tests suggest that the relationship between house prices and other variables, 
such as population, new building permits, GDP, and the number of employed individuals, 
may be more complex and cannot be explained solely through linear causality. Therefore, a more comprehensive analysis that 
incorporates additional factors and non-linear relationships may be necessary to better understand the dynamics of the Italian 
housing market.
\section*{Bibliography}
\label{section:Bibliography}
    \href{https://fred.stlouisfed.org/series/QITR628BIS}{House prices}
    \href{https://infostat.bancaditalia.it/inquiry/home?spyglass/taxo:CUBESET=/PRINC_IND_00/PRINC_IND_01/PRINC_IND_02/PRINC_IND_02_02&ITEMSELEZ=BAM_MIR.M.1300010.MIR5411.9.950.1000.SBI17.EUR.110.212:true&OPEN=false/&ep:LC=IT&COMM=BANKITALIA&ENV=LIVE&CTX=DIFF&IDX=1&/view:CUBEIDS=BAM_MIR.M.1300010.MIR5411.9.950.1000.SBI17.EUR.110.212} {Interest Rate} 
    \href{http://dati.istat.it/#}{Employees} 
    \href{https://ourworldindata.org/grapher/population}{Population} \\
    \href{http://dati.istat.it/#}{Building Permits} 
    \href{http://dati.istat.it/Index.aspx?DataSetCode=DCCN_PILT}{DCP}
\end{document}